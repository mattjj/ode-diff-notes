\documentclass{article}
\usepackage{amsmath}
\usepackage{amssymb}

\begin{document}
\section{Jacobian-vector products}
Consider the parameterized ODE initial value problem
\begin{equation}
    \dot y = f(t, y, a), \qquad y(0, a) = y_0(a),
\end{equation}
by which we mean
\begin{equation}
    \partial_0 y(t, a) = f(t, y(t, a), a). \qquad y(0, a) = y_0(a).
\end{equation}
We want to understand how the solution to the ODE changes (e.g. at particular
values of $t$) for small perturbations of $a$. That is, we want to compute the
Jacobian-vector product
\begin{equation}
    (a, v) \mapsto \partial_1 y(t, a)[v]
\end{equation}
where $v$ is a small perturbation to $a$.

Since the ODE holds true for all values of $a$ (or at least those close to a
particular $a_0$ in which we are interested), we can view both sides as
functions of $a$, and assuming differentiability we can differentiate both
sides with respect to $a$ to find a new equation that must be satisfied:
\begin{equation}
    \partial_1 \partial_0 y(t, a) = \partial_2 f(t, y(t, a), a) + \partial_1 f(t, y(t, a), a) \circ \partial_1 y(t, a).
\end{equation}
Using the fact that partial derivatives commute, we can identify $z(t, a)
\triangleq \partial_1 y(t, a)[v]$ as a new state vector to write a joint ODE system
\begin{gather}
    \begin{bmatrix}
        \dot y \\
        \dot z
    \end{bmatrix}
    =
    \begin{bmatrix}
        f(t, y, a) \\
        g(t, y, z, a)
    \end{bmatrix},
    \qquad
    \begin{bmatrix}
        y(0, a) \\
        z(0, a)
    \end{bmatrix}
    =
    \begin{bmatrix}
        y_0(a) \\
        \partial y_0(a)[v]
    \end{bmatrix},
    \\
    g(t, y, z, a) = \partial_1 f(t, y, a)[z] + \partial_2 f(t, y, a)[v].
\end{gather}
Notice that the dynamics on the $z$ component are linear/affine in $z$ (and $v$!).

\section{Vector-Jacobian products}
Consider the parameterized linear/affine ODE IVP
\begin{equation}
    \partial_0 z(t) = A(t) z(t) + B(t) v, \qquad z(0) = C v.
\end{equation}
On the vector/affine space of solutions $\mathcal{Y}$ consider a nice linear
functional $\mathcal{D}: \mathcal{Y} \to \mathbb{R}$. This induces a linear
function on the vector space of possible perturbations $v$.


\end{document}

