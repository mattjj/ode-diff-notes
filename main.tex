\documentclass{article}
\usepackage{amsmath}
\usepackage{amssymb}

\begin{document}
\section{Jacobian-vector products}
Consider the parameterized ODE initial value problem
\begin{equation}
    \dot y = f(t, y, a), \qquad y(0, a) = y_0(a),
\end{equation}
by which we mean
\begin{equation}
    \partial_0 y(t, a)[1] = f(t, y(t, a), a), \qquad y(0, a) = y_0(a),
\end{equation}
for all $t$ and $a$ in some domains.
We want to understand how the solution to the ODE changes (e.g. at particular
values of $t$) for small perturbations of $a$. That is, we want to be able to
compute the Jacobian-vector product
\begin{equation}
    (a, v) \mapsto \partial_1 y(t, a)[v]
\end{equation}
at any particular values of $t$ and $a$, where $v$ can be interpreted as a
small perturbation to the value of $a$.

Since the ODE holds true for all values of $a$ (or at least those close to a
particular $a_0$ in which we are interested), we can view both sides as
functions of $a$, and assuming differentiability we can differentiate both
sides with respect to $a$ to find a new equation that must be satisfied:
\begin{equation}
    \partial_1 (\, (t, a) \mapsto \partial_0 y(t, a)[1] \,) = \partial_2 f(t, y(t, a), a) + \partial_1 f(t, y(t, a), a) \circ \partial_1 y(t, a).
\end{equation}
Applying both sides to a particular perturbation vector $v$ and using the fact
that partial derivaives commute, we have
\begin{equation}
    \partial_0 (\, (t, a) \mapsto \partial_1 y(t, a)[v] \,)[1] = \partial_2 f(t, y(t, a), a)[v] + \partial_1 f(t, y(t, a), a) [ \, \partial_1 y(t, a)[v] \,].
    \notag
\end{equation}
We can identify $z(t, a) \triangleq \partial_1 y(t, a)[v]$ as a new state
vector to write a joint ODE system
\begin{gather}
    \begin{bmatrix}
        \dot y \\
        \dot z
    \end{bmatrix}
    =
    \begin{bmatrix}
        f(t, y, a) \\
        g(t, y, z, a)
    \end{bmatrix},
    \qquad
    \begin{bmatrix}
        y(0, a) \\
        z(0, a)
    \end{bmatrix}
    =
    \begin{bmatrix}
        y_0(a) \\
        \partial y_0(a)[v]
    \end{bmatrix},
    \\
    g(t, y, z, a) = \partial_1 f(t, y, a)[z] + \partial_2 f(t, y, a)[v].
\end{gather}
Notice that the dynamics on the $z$ component are linear/affine in $z$ (and $v$!).

\section{Transposing linear ODEs}
Consider the parameterized linear ODE IVP
\begin{equation}
    \dot z(t) = A(t) z(t) + B(t) v, \qquad z(0) = C v,
    \label{eq:linear_ode}
\end{equation}
as a function of $v$.
The implicit mapping $\mathcal{T}_1: v \mapsto z$ is linear, and so for any
linear functional on solution functions $\mathcal{T}_2 : z \mapsto \mathbb{R}$
there is linear function on perturbations $v$ defined by $\mathcal{T}_2 \circ
\mathcal{T}_1 : v \mapsto \mathbb{R}$. Given a representer for such a linear
functional $\mathcal{T}_2$, we wish to find an explicit representer vector for
$\mathcal{T}_2 \circ \mathcal{T}_1$.

% Concretely, consider linear functionals of the form
% \begin{equation}
%     \mathcal{T}_2[z] = w_T^\mathsf{T} z(T) + \int_0^T w(t)^\mathsf{T} z(t) \, \mathrm{d}t.
%     \label{eq:functional}
% \end{equation}
% We wish to find a function $\lambda(t)$ such that
% \begin{equation}
%     w_T^\mathsf{T} z(T) + \int_0^T w(t)^\mathsf{T} z(t) \, \mathrm{d}t
%     =
%     \lambda(0)^\mathsf{T} C v + \int_0^T \lambda(t)^\mathsf{T} B(t)[v] \, \mathrm{d}t,
% \end{equation}
% for all $v$, when $z$ is a solution to~\eqref{eq:linear_ode}.

Consider first the special case when $B \equiv 0$, so that we have the ODE
\begin{equation}
    \dot z(t) = A(t) z(t), \qquad z(0) = C v.
    \label{eq:linear_ode_simple}
\end{equation}
Moreover consider the special case of a weighted evaluation functional
\begin{equation}
    \mathcal{T}_2[z] = w^\mathsf{T} z(T).
    \label{eq:evaluation_functional}
\end{equation}
We wish to find a representer vector $\lambda$ such that
\begin{equation}
\lambda^\mathsf{T} z(0) = w^\mathsf{T} z(T).
\end{equation}
Since the particular time $t=0$ is arbitrary, a more general problem would be
to find a function $\lambda(t)$ such that
\begin{equation}
    \lambda(t)^\mathsf{T} z(t) = w^\mathsf{T} z(T)
    \label{eq:representer_simple}
\end{equation}
for all times $t$.
That is, we can fix $\lambda(T) = w$ and ensure that the value of
$\lambda(t)^\mathsf{T} z(t)$ does not change with time:
\begin{align}
    0 &= \partial (\, t \mapsto \lambda(t)^\mathsf{T} z(t) \,)
    = \dot \lambda(t)^\mathsf{T} z(t) + \lambda(t)^\mathsf{T} \dot z(t)
    \\
    &= \dot \lambda(t)^\mathsf{T} z(t) + \lambda(t)^\mathsf{T} A(t) z(t),
\end{align}
where on the last line we have used the ODE~\eqref{eq:linear_ode_simple}.
To satisfy this equation for all $t$ and arbitrary solutions $z(t)$, we can
choose
\begin{equation}
    \dot \lambda(t) = - A(t)^\mathsf{T} \lambda(t), \qquad \lambda(T) = w.
    \label{eq:adjoint_ode_simple}
\end{equation}
This gives us a means of computing a representer for the linear functional
$\mathcal{T}_2 \circ \mathcal{T}_1$ by solving the ODE
IVP~\eqref{eq:adjoint_ode_simple}, integrating backward in time from $t=T$ to
$t=0$ to compute $\lambda(0)$.
Notice that by linearity we can handle a functional that is a linear
combination of such weighted evaluation functionals, say at times $0 < T_1 <
T_2$, by pulling back the functional at time $t=T_2$ to a representer at time
$t=T_1$ and summing before pulling back the sum to $t=0$.

We can follow a similar argument when $B \not \equiv 0$. For a $z$
solving~\eqref{eq:linear_ode} and a linear functional of the
form~\eqref{eq:evaluation_functional}, we can seek a function $\lambda(t)$
that represents the linear function by satisfying
\begin{equation}
    w^\mathsf{T} z(T) = \lambda(t)^\mathsf{T} z(t) - \int_T^t \lambda(\tau)^\mathsf{T} B(\tau) v \, \mathrm{d} \tau,
    \label{eq:representer}
\end{equation}
for all times $t$. In particular, at time $t=0$ we would have
\begin{equation}
    w^\mathsf{T} z(T) = \lambda(0)^\mathsf{T} z(0) - \int_T^0 \lambda(\tau)^\mathsf{T} B(\tau) v \, \mathrm{d} \tau,
\end{equation}
which gives us a representer of $\mathcal{T}_2 \circ \mathcal{T}_1$, namely as
\begin{equation}
    (\mathcal{T}_2 \circ \mathcal{T}_1)[v] = u^\mathsf{T} v
    \quad \text{where} \quad
    u = \lambda(0) - \int_T^0 B(\tau)^\mathsf{T} \lambda(\tau) \, \mathrm{d} \tau.
    \label{eq:representer2}
\end{equation}
We can find an ODE which $\lambda(t)$ must satisfy by
differentiating both sides of~\eqref{eq:representer} with respect to time:
\begin{align}
    0 &= \lambda(t)^\mathsf{T} \dot z(t) + \dot \lambda(t)^\mathsf{T} z(t) - \lambda(t)^\mathsf{T} B(t) v
    \\
    &=
    \lambda(t)^\mathsf{T} \left( A(t) z(t) + B(t) v \right) + \dot \lambda(t)^\mathsf{T} z(t) - \lambda(t)^\mathsf{T} B(t) v
    \\
    &= \lambda(t)^\mathsf{T} A(t) z(t) + \dot \lambda(t)^\mathsf{T} z(t),
\end{align}
and so as before $\lambda(t)$ must satisfy the linear ODE IVP
\begin{equation}
    \dot \lambda(t) = -A(t)^\mathsf{T} \lambda(t), \qquad \lambda(T) = w.
\end{equation}
To compute the integral in~\eqref{eq:representer2}, we can augment the ODE
system to
\begin{equation}
    \begin{bmatrix}
        \dot \lambda \\
        \dot \omega
    \end{bmatrix}
    =
    \begin{bmatrix}
        - A(t)^\mathsf{T} \lambda(t)
        \\
        B(t)^\mathsf{T} \lambda(t)
    \end{bmatrix},
    \qquad
    \begin{bmatrix}
        \lambda(T) \\
        \omega(T)
    \end{bmatrix}
    =
    \begin{bmatrix}
        w \\
        0
    \end{bmatrix}.
\end{equation}

\end{document}
